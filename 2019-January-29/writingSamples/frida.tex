\documentclass[]{article}
\usepackage{lmodern}
\usepackage{amssymb,amsmath}
\usepackage{ifxetex,ifluatex}
\usepackage{fixltx2e} % provides \textsubscript
\ifnum 0\ifxetex 1\fi\ifluatex 1\fi=0 % if pdftex
  \usepackage[T1]{fontenc}
  \usepackage[utf8]{inputenc}
\else % if luatex or xelatex
  \ifxetex
    \usepackage{mathspec}
  \else
    \usepackage{fontspec}
  \fi
  \defaultfontfeatures{Ligatures=TeX,Scale=MatchLowercase}
\fi
% use upquote if available, for straight quotes in verbatim environments
\IfFileExists{upquote.sty}{\usepackage{upquote}}{}
% use microtype if available
\IfFileExists{microtype.sty}{%
\usepackage[]{microtype}
\UseMicrotypeSet[protrusion]{basicmath} % disable protrusion for tt fonts
}{}
\PassOptionsToPackage{hyphens}{url} % url is loaded by hyperref
\usepackage[unicode=true]{hyperref}
\hypersetup{
            pdfborder={0 0 0},
            breaklinks=true}
\urlstyle{same}  % don't use monospace font for urls
\IfFileExists{parskip.sty}{%
\usepackage{parskip}
}{% else
\setlength{\parindent}{0pt}
\setlength{\parskip}{6pt plus 2pt minus 1pt}
}
\setlength{\emergencystretch}{3em}  % prevent overfull lines
\providecommand{\tightlist}{%
  \setlength{\itemsep}{0pt}\setlength{\parskip}{0pt}}
\setcounter{secnumdepth}{0}
% Redefines (sub)paragraphs to behave more like sections
\ifx\paragraph\undefined\else
\let\oldparagraph\paragraph
\renewcommand{\paragraph}[1]{\oldparagraph{#1}\mbox{}}
\fi
\ifx\subparagraph\undefined\else
\let\oldsubparagraph\subparagraph
\renewcommand{\subparagraph}[1]{\oldsubparagraph{#1}\mbox{}}
\fi

% set default figure placement to htbp
\makeatletter
\def\fps@figure{htbp}
\makeatother


\date{}

\begin{document}

\textbf{Note to reader}: This is the introduction of a manuscript titled
``Time of day and a ketogenic diet influence susceptibility to SUDEP in
\emph{Scn1a \textsuperscript{R1407X/+}} mice''. This was submitted but
will likely bounce back with a bunch of reviewers' comments. I wanted to
take this opportunity to revisit my introduction and make sure it makes
sense to the reader in terms of content and flow. I'm only including the
abstract here for those that want more context, so don't feel like you
have to revise it. Thank you for your thoughtful feedback!

\textbf{Abstract}

Sudden unexpected death in epilepsy (SUDEP) is a major cause of
mortality in patients with drug-resistant epilepsy. Most SUDEP cases
occur in bed at night and are preceded by a generalized tonic-clonic
seizure (GTCS) and. Dravet syndrome (DS) is a severe childhood-onset
epilepsy commonly caused by mutations in the \emph{SCN1A} gene. Affected
individuals suffer from refractory seizures and an increased risk of
SUDEP. Here, we demonstrate that mice with the
\emph{Scn1a\textsuperscript{R1407X/+}} loss-of-function mutation
experience more spontaneous seizures and SUDEP during the early night.
In DS mice we evaluate effects of long-term ketogenic diet (KD)
treatment on mortality and seizure frequency. DS mice showed high
premature mortality (44\% survival by P60) that was associated with
increased spontaneous GTCSs 1-2 days prior to death. KD treated mice had
a significant reduction in mortality (86\% survival by P60) compared to
mice fed a control diet. Interestingly, increased survival was not
associated with decreased spontaneous non-fatal seizures. Further
studies are needed to determine how KD confers protection from SUDEP.
Moreover, our findings implicate time of day as a factor influencing the
occurrence of seizures and SUDEP. DS mice, though nocturnal, are more
likely to have SUDEP at night, suggesting that the increased incidence
of SUDEP at night in DS mice may not be solely due to sleep.

\textbf{INTRODUCTION}

Sudden unexpected death in epilepsy (SUDEP) is estimated to occur in
\textasciitilde{}17\% of patients with epilepsy\textsuperscript{1}. This
number can drastically increase to 50\% in patients with poorly
controlled and severe epilepsy\textsuperscript{2}. Although the
mechanisms underlying SUDEP are not fully understood, an increasing body
of evidence suggests SUDEP is due to seizure-induced cardiorespiratory
dysfunction\textsuperscript{3,4}. However, little is known about the
circumstances leading up to SUDEP. A strong association with sleep has
been documented in a number of studies\textsuperscript{3,5}. Although a
significant majority of patients are found in bed in the prone position
at the time of death\textsuperscript{6-8}, the occurrence of SUDEP
during sleep varies widely among published case
studies\textsuperscript{9}. This suggests that other circadian factors
may be relevant.

Dravet Syndrome (DS) is a devastating epileptic encephalopathy of
childhood-onset that typically manifests as febrile seizures in the
first year of life and progresses to refractory
epilepsy\textsuperscript{10}. Children with DS develop several
comorbidities, such as ataxia, sleep disturbance and cognitive
impairments\textsuperscript{11}. In patients with DS, the risk of SUDEP
is estimated to be 15 times higher than in other pediatric epilepsies.
Premature death occurs in 21\% of DS patients, with SUDEP accounting for
nearly half of these deaths\textsuperscript{12}. Approximately 85\% of
DS cases are caused by dominant loss-of-function mutations of the
\emph{SCN1A} gene, which encodes the neuronal voltage-gated sodium
channel Nav1.1\textsuperscript{11,13}. DS mouse models have proven to be
an efficient research tool for understanding the pathophysiology of
SUDEP as they recapitulate many aspects of the clinical condition: they
have spontaneous seizures and a high incidence of premature
mortality\textsuperscript{14}. They also display impaired sleep
architecture homeostasis\textsuperscript{10}.

A recent study found that time of day can have an independent influence
on physiological changes associated with a seizure, particularly
breathing\textsuperscript{15}. This is important as seizure-induced
changes in respiratory physiology contribute to SUDEP in
patients\textsuperscript{3,16-22} and in DS mice\textsuperscript{14}. In
the present study, we aimed to determine in DS mice whether: 1)
spontaneous seizures and SUDEP are more likely to occur in the light or
dark phase; 2) seizure frequency changes in the days prior to SUDEP; and
3) treatment with a high-fat, low-carbohydrate ketogenic diet (KD),
which has been proven to be protective in other seizure
models\textsuperscript{23,24}, results in fewer spontaneous seizures and
death.

\end{document}
