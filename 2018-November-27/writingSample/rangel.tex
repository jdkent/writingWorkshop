\documentclass[]{article}
\usepackage{lmodern}
\usepackage{amssymb,amsmath}
\usepackage{ifxetex,ifluatex}
\usepackage{fixltx2e} % provides \textsubscript
\ifnum 0\ifxetex 1\fi\ifluatex 1\fi=0 % if pdftex
  \usepackage[T1]{fontenc}
  \usepackage[utf8]{inputenc}
\else % if luatex or xelatex
  \ifxetex
    \usepackage{mathspec}
  \else
    \usepackage{fontspec}
  \fi
  \defaultfontfeatures{Ligatures=TeX,Scale=MatchLowercase}
\fi
% use upquote if available, for straight quotes in verbatim environments
\IfFileExists{upquote.sty}{\usepackage{upquote}}{}
% use microtype if available
\IfFileExists{microtype.sty}{%
\usepackage[]{microtype}
\UseMicrotypeSet[protrusion]{basicmath} % disable protrusion for tt fonts
}{}
\PassOptionsToPackage{hyphens}{url} % url is loaded by hyperref
\usepackage[unicode=true]{hyperref}
\hypersetup{
            pdfborder={0 0 0},
            breaklinks=true}
\urlstyle{same}  % don't use monospace font for urls
\usepackage{graphicx,grffile}
\makeatletter
\def\maxwidth{\ifdim\Gin@nat@width>\linewidth\linewidth\else\Gin@nat@width\fi}
\def\maxheight{\ifdim\Gin@nat@height>\textheight\textheight\else\Gin@nat@height\fi}
\makeatother
% Scale images if necessary, so that they will not overflow the page
% margins by default, and it is still possible to overwrite the defaults
% using explicit options in \includegraphics[width, height, ...]{}
\setkeys{Gin}{width=\maxwidth,height=\maxheight,keepaspectratio}
\IfFileExists{parskip.sty}{%
\usepackage{parskip}
}{% else
\setlength{\parindent}{0pt}
\setlength{\parskip}{6pt plus 2pt minus 1pt}
}
\setlength{\emergencystretch}{3em}  % prevent overfull lines
\providecommand{\tightlist}{%
  \setlength{\itemsep}{0pt}\setlength{\parskip}{0pt}}
\setcounter{secnumdepth}{0}
% Redefines (sub)paragraphs to behave more like sections
\ifx\paragraph\undefined\else
\let\oldparagraph\paragraph
\renewcommand{\paragraph}[1]{\oldparagraph{#1}\mbox{}}
\fi
\ifx\subparagraph\undefined\else
\let\oldsubparagraph\subparagraph
\renewcommand{\subparagraph}[1]{\oldsubparagraph{#1}\mbox{}}
\fi

% set default figure placement to htbp
\makeatletter
\def\fps@figure{htbp}
\makeatother


\date{}

\begin{document}

\textbf{Here, I tried to convey the basic need for finding new methods
of measuring inhibition in human subjects, other than TMS. If it feels
somewhat congested, this is because I wanted to challenge myself and see
if could fit a full idea of\ldots{}.}

\begin{enumerate}
\def\labelenumi{\arabic{enumi}.}
\item
  \textbf{TMS as a tool for measuring Inhibition}
\item
  \textbf{Problems with this method}
\item
  \textbf{Possible new method}
\end{enumerate}

\textbf{\ldots{}. in the 500 words allowed. I hope that you will take
this as somewhat of a ``break'' from the more rigorous, and frankly
better, submissions we have read thus far. }

\textbf{My main concern is how well I was able to communicate the above
ideas, in a way that comes off as summary, but is also understandable
and not to scatter-shot. I suppose it could be read like an Introduction
to a longer paper, so any suggestions on making it feel more like that,
would be swell!! }

\textbf{I have included figures as a visual aid. }

\textbf{See you all soon!}

\textbf{\emph{Looking towards non-TMS methods of Cortical-Spinal
Inhibition}}

Transcranial Magnetic Stimulation (TMS) provides relatively focal and
well tolerated stimulations of neuronal populations, while remaining
non-invasive. When applied over primary motor cortex (M1), TMS produces
a corresponding spike in muscular activity known as a Motor Evoked
Potential (MEP, Figure 1.). If the cortical neurons responsible for the
movement of that muscle are experiencing inhibition (GABAergic) at the
time of the TMS pulse, their neurotransmitter release will be shunted,
and the MEP will have decreased amplitude (Cortico-Spinal Inhibition,
Figure 2.). TMS remains as one of the primary tools for the measurement
of inhibition at the cortical level, with applications for pharmacology,
physical therapy and cognitive studies.

However, TMS presents both experimental issues and unexplained phenomena
that must be taken into account during research. TMS coil direction
alone has been hypothesized to preferentially activate different
populations of M1 interneurons, resulting in different effects on the
MEP (Figure 3.). Alternate coils (H-Coils), have been devised to reduce
any pulse-to-pulse variability in position due to experimenter error,
but they are so far largely restricted to clinical settings.
Additionally, stands used to hold the TMS coil to the head of the
participant can also eliminate placement variability, but require
immobility on the part of the participant, which can limit the
flexibility of behavioral tasks. Ultimately, even when the TMS coil
position is inert, MEP amplitudes are known to be vary pulse-to-pulse
and can be suppressed by the TMS pulses themselves over the course of a
longer experiment. TMS pulses also produce loud sounds and haptic
sensations on the skull, which can become distracting and/or startling
at higher intensities, and activate other, non-target, muscles. Finally,
patients with epilepsy or traumatic brain injury are often excluded from
TMS studies as a precaution, which eliminates the prospect of studying
inhibition in such a unique population. Human studies looking at
Cortico-Spinal inhibition with TMS must be designed such that they
accommodate these physiological drawbacks. While TMS still remains a
wonderful method for many task paradigms, it becomes particularly
troublesome in tasks where small effect sizes may be susceptible to MEP
variability, the patient population cannot use TMS, or where
loud/jarring TMS pulses could alter the inhibitive responses.

The challenges of using TMS to measure inhibition have prompted the
investigation into more universal methods. Some recent research has
focused on correlating Electroencephalography (EEG) signatures
(fronto-central P3), with decreased force output and/or reduced muscular
activity. For example, in tasks where inhibition promoting (unexpected)
stimuli are presented, participants can be instructed to hold a force
meter at a constant level (Figure 4.). Previous findings have shown that
both EEG signatures and force output of the hand show distinct
modulations following unexpected stimuli, which are consistent with an
inhibitory process (Figure 5.). Further investigations are currently
underway to evaluate if the EEG/Force method of measuring Cortico-Spinal
inhibition can viably substitute for TMS, during cognitive tasks. If so,
many of the previously unavailable task paradigms and patient
populations will be open for use in probing the questions regarding
inhibition.

\end{document}
