\documentclass[]{article}
\usepackage{lmodern}
\usepackage{amssymb,amsmath}
\usepackage{ifxetex,ifluatex}
\usepackage{fixltx2e} % provides \textsubscript
\ifnum 0\ifxetex 1\fi\ifluatex 1\fi=0 % if pdftex
  \usepackage[T1]{fontenc}
  \usepackage[utf8]{inputenc}
\else % if luatex or xelatex
  \ifxetex
    \usepackage{mathspec}
  \else
    \usepackage{fontspec}
  \fi
  \defaultfontfeatures{Ligatures=TeX,Scale=MatchLowercase}
\fi
% use upquote if available, for straight quotes in verbatim environments
\IfFileExists{upquote.sty}{\usepackage{upquote}}{}
% use microtype if available
\IfFileExists{microtype.sty}{%
\usepackage[]{microtype}
\UseMicrotypeSet[protrusion]{basicmath} % disable protrusion for tt fonts
}{}
\PassOptionsToPackage{hyphens}{url} % url is loaded by hyperref
\usepackage[unicode=true]{hyperref}
\hypersetup{
            pdfborder={0 0 0},
            breaklinks=true}
\urlstyle{same}  % don't use monospace font for urls
\IfFileExists{parskip.sty}{%
\usepackage{parskip}
}{% else
\setlength{\parindent}{0pt}
\setlength{\parskip}{6pt plus 2pt minus 1pt}
}
\setlength{\emergencystretch}{3em}  % prevent overfull lines
\providecommand{\tightlist}{%
  \setlength{\itemsep}{0pt}\setlength{\parskip}{0pt}}
\setcounter{secnumdepth}{0}
% Redefines (sub)paragraphs to behave more like sections
\ifx\paragraph\undefined\else
\let\oldparagraph\paragraph
\renewcommand{\paragraph}[1]{\oldparagraph{#1}\mbox{}}
\fi
\ifx\subparagraph\undefined\else
\let\oldsubparagraph\subparagraph
\renewcommand{\subparagraph}[1]{\oldsubparagraph{#1}\mbox{}}
\fi

% set default figure placement to htbp
\makeatletter
\def\fps@figure{htbp}
\makeatother


\date{}

\begin{document}

Gail Harmata

Writing Sample for WIN

September 11, 2018

MEMO: I have a lot of points I would like to hit in the discussion of my
paper, but I'm not sure the best way to start. The paper is about how we
stimulated the brain in three patients with epilepsy and saw unusually
long-lasting effects on breathing under certain conditions. We did more
detailed work regarding the breathing, the nuances of the effect, and
quantifying the brain---all of which I would like to discuss---but the
most important point is that this can occur at all. \emph{\emph{I would
like feedback on how this start to my discussion section reads
currently, and thoughts regarding order of content in a discussion
section. }}

\emph{P.S. SUDEP = Sudden Unexpected Death in Epilepsy (already
explained earlier in the manuscript)}

\emph{Discussion}

Here we report that electrical stimulation and seizure within the
% change is able to cause to causes (stronger claim)
% end sentence after breathing
amygdala causes apnea and other abnormalities of breathing.
% New sentence, start with something catchy (like remarkably!)
% use apnea instead of hypoventilation (or vice versa) 
% (or are the two words describing unique properties?)
Remarkably, apnea can persist for minutes even after
% Instead of "In previously", you could say, 
% "This stands in stark contrast to previous reports"
stimulation or seizure in the amygdala has ceased. 
% I don't quite like my suggested sentence either, but maybe it'll be useful.
This stands in stark contrast to previous reports where normal breathing
resumed within 15 seconds after amygdala stimulation\textsuperscript{1-3}. This
% I really like this last sentence.
novel finding of long-lasting breathing changes may have critical
implications for understanding postictal mortality.

\emph{Prevalence of the prolonged disrupted breathing phenotype}

% looks good
It is unclear what proportion of epilepsy patients show prolonged
postictal or post-stimulation hypoventilation, and how predictive this
% looks good
is of future SUDEP. It has recently been reported that periictal apnea
is common, especially in temporal lobe seizures, but that prolonged
% looks good
postictal apnea is far less common\textsuperscript{4}. Similarly,
post-stimulation breathing changes are not always observed in our work,
as we have studied numerous (18) other patients who did not show
pronounced, long-lasting hypoventilation after the end of amygdala
% looks good
stimulation. However, we did note that the patients who did have
prolonged breathing alterations (P352, P384, and P413) all had seizure
% Thus, it... may be better as two/three sentences
foci involving the temporal lobe. Thus, it is possible that prolonged
breathing disruption following stimulation or seizure could represent a
biomarker specific to a subset of patients. One such biomarker could 
be abnormal connectivity of the mesial temporal lobe after years of 
uncontrolled seizures. Proper connectivity may be important for breathing
to remain disrupted after the end of amygdala stimulation or seizure.

%looks good
However, it is possible that the prevalence of prolonged breathing
% looks good
disruption is under-recognized. For seizures observed in epilepsy
monitoring units, a typical clinical approach is to intervene by
immediately asking the patient follow-up questions and performing a
% looks good
neurological exam. This would disrupt breathing signal immediately
following the seizure and potentially mask changes by causing the
% looks good
patient to voluntarily breathe in order to talk. Thus, it is possible
that postictal apnea or irregular breathing is more common than
% looks good
anticipated. We were able to gather valuable postictal data for P413 by
adapting the intervention protocol for a partial seizure by limiting
disruption of natural breathing patterns. Use of this approach when
possible could reveal more instances of postictal breathing disruption
% looks good (maybe split into two sentences...)
than currently detected, and may be critical for understanding SUDEP
events that occur when the patient is unlikely to be prompted to talk
(alone, at night, in bed, etc.).

% looks good
It is also possible that more patients may exhibit prolonged breathing
disruption with amygdala stimulation, but were not detected due to
% looks good
variations in electrode placement. For the patients we describe in this
report, site of amygdala stimulation was crucial in whether or not apnea
occurred at all, and whether or not apnea persisted after stimulation
% looks good
ceased. Thus, it is possible that stimulation or seizure must occur in a
specific location to cause acute and/or prolonged effects, even within
% looks good
the same patient. It is therefore possible that more patients with
prolonged breathing changes could have been detected in our stimulation
studies had their electrodes been placed in a slightly different
location.

References

\textsuperscript{1} Dlouhy, B. J. \emph{et al.} Breathing Inhibited When
Seizures Spread to the Amygdala and upon Amygdala Stimulation. \emph{The
Journal of Neuroscience} \textbf{35}, 10281-10289,
doi:10.1523/jneurosci.0888-15.2015 (2015).

\textsuperscript{2} Lacuey, N., Zonjy, B., Londono, L. \& Lhatoo, S. D.
Amygdala and hippocampus are symptomatogenic zones for central apneic
seizures. \emph{Neurology} \textbf{88}, 701-705,
doi:10.1212/wnl.0000000000003613 (2017).

\textsuperscript{3} Nobis, W. P. \emph{et al.}
Amygdala-stimulation-induced apnea is attention and nasal-breathing
dependent. \emph{Annals of Neurology}, n/a-n/a, doi:10.1002/ana.25178
(2018).

\textsuperscript{4} Lacuey, N. \emph{et al.} The incidence and
significance of periictal apnea in epileptic seizures. \emph{Epilepsia},
doi:10.1111/epi.14006 (2018).

\end{document}
