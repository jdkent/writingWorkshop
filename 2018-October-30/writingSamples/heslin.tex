\documentclass[]{article}
\usepackage{lmodern}
\usepackage{amssymb,amsmath}
\usepackage{ifxetex,ifluatex}
\usepackage{fixltx2e} % provides \textsubscript
\ifnum 0\ifxetex 1\fi\ifluatex 1\fi=0 % if pdftex
  \usepackage[T1]{fontenc}
  \usepackage[utf8]{inputenc}
\else % if luatex or xelatex
  \ifxetex
    \usepackage{mathspec}
  \else
    \usepackage{fontspec}
  \fi
  \defaultfontfeatures{Ligatures=TeX,Scale=MatchLowercase}
\fi
% use upquote if available, for straight quotes in verbatim environments
\IfFileExists{upquote.sty}{\usepackage{upquote}}{}
% use microtype if available
\IfFileExists{microtype.sty}{%
\usepackage[]{microtype}
\UseMicrotypeSet[protrusion]{basicmath} % disable protrusion for tt fonts
}{}
\PassOptionsToPackage{hyphens}{url} % url is loaded by hyperref
\usepackage[unicode=true]{hyperref}
\hypersetup{
            pdfborder={0 0 0},
            breaklinks=true}
\urlstyle{same}  % don't use monospace font for urls
\IfFileExists{parskip.sty}{%
\usepackage{parskip}
}{% else
\setlength{\parindent}{0pt}
\setlength{\parskip}{6pt plus 2pt minus 1pt}
}
\setlength{\emergencystretch}{3em}  % prevent overfull lines
\providecommand{\tightlist}{%
  \setlength{\itemsep}{0pt}\setlength{\parskip}{0pt}}
\setcounter{secnumdepth}{0}
% Redefines (sub)paragraphs to behave more like sections
\ifx\paragraph\undefined\else
\let\oldparagraph\paragraph
\renewcommand{\paragraph}[1]{\oldparagraph{#1}\mbox{}}
\fi
\ifx\subparagraph\undefined\else
\let\oldsubparagraph\subparagraph
\renewcommand{\subparagraph}[1]{\oldsubparagraph{#1}\mbox{}}
\fi

% set default figure placement to htbp
\makeatletter
\def\fps@figure{htbp}
\makeatother


\date{}

\begin{document}

Background Info:

My WIN submission this week is the Specific Aims portion of an NRSA
(F31) that I will be resubmitting in either December or April, depending
on the reviewer comments I receive. This portion of the F31 must not
exceed 1 page, so feedback on balancing clarity and brevity would be
great.

\textbf{Specific Aims}

Cognitive dysfunction is a core feature of schizophrenia as it is often
present prior to diagnosis and is a strong predictor for quality of
life\textsuperscript{1--3}. Dysfunctional neuronal activity in the
prefrontal cortex has been associated with impaired cognitive functions
(e.g., attention, working memory, thought coordination) commonly
observed in this disease\textsuperscript{4--6}. However, current
treatments do not improve cognitive symptoms in schizophrenia.
Therefore, to develop new therapies for these deficits, there is a
\emph{critical need} to investigate the neural circuitry underlying
cognitive symptoms of schizophrenia.

The cerebellum is largely associated with coordinating motor functions,
however, relayed cerebellar projections to cortical areas like the
prefrontal cortex are instrumental in a variety of cognitive and
executive functions\textsuperscript{7,8}. In line with these
observations, one neuronal pathway implicated in the cognitive symptoms
of schizophrenia is a cerebello-thalamo-cortico- cerebellar
loop\textsuperscript{9}. Decreased structural and functional
connectivity within this pathway and dysfunction of its nodes has been
observed in patients with schizophrenia\textsuperscript{10--12}. It is
hypothesized that the role of the cerebellum in this circuit is to
coordinate cognitive processing through modulation of areas like the
prefrontal cortex\textsuperscript{13,14}. The dentate nuclei, or lateral
cerebellar nuclei (LCN) in rodents, send the majority of cerebellar
input to the prefrontal cortex\textsuperscript{15,16}. A population of
inhibitory D1 dopamine receptor (D1DR)-expressing neurons in the LCN has
been recently implicated in pre-frontal dependent cognitive tasks
\textsuperscript{17}. However, precisely how D1DR-expressing cells of
the LCN contribute to cognitive function and modulation of the frontal
cortex is unclear.

To address this, we will evaluate the role of LCN D1DR-expressing
neurons in performance of an interval timing task. In interval timing
tasks, subjects are presented with a cue that signals them to respond
after a certain duration elapses (e.g., press a lever after 12 seconds).
Interval timing performance is impaired in schizophrenia
patients\textsuperscript{18--21}, depends on multiple nodes in our
circuit of interest\textsuperscript{22,23}, and is impacted by
manipulations of dopaminergic signaling\textsuperscript{24--26}.
Successful interval timing performance requires recruitment and
coordination of several different executive functions such as working
memory, attention, and decision making. Importantly, interval timing
task measures can be readily compared across animal models,
neuropsychiatric patients, and healthy human comparisons. Therefore, it
is an ideal type of task with which to probe our circuit and neuronal
population of interest in rodent models. Specifically, we will
investigate how LCN D1DR-expressing neurons contribute to performance in
a peak interval timing task, and how performance correlates with
population activity in the LCN and prelimbic cortex (PL), the rodent
medial frontal cortex. Based on our prior work and the primary
literature, our \textbf{\emph{overall hypothesis}} is that inhibitory
D1DR-expressing cells of the LCN are necessary for successful interval
timing through modulation of activity in the PL.

\textbf{\emph{Aim 1. Determine if LCN D1DR-expressing neurons are
necessary for interval timing}}

The cerebellum and LCN specifically have been implicated in both
sub-second and supra-second interval timing\textsuperscript{23,27--30}.
Dopaminergic signaling is crucial for timing behavior and is one of
several abnormal neurotransmitter systems in
schizophrenia\textsuperscript{4,21}. The role of inhibitory
D1DR-expressing neurons in the LCN is presently unknown. We will
determine if LCN D1DR-expressing neurons are necessary for interval
timing performance using optogenetically driven inactivation during the
task. Additionally, synchronization of cortical pyramidal cell activity
by inhibitory interneurons is associated with cognitive
processing\textsuperscript{31,32}. Therefore, we will record local field
potentials (LFPs) and single neurons during optogenetic inactivation to
assess whether LCN D1DR-expressing neurons contribute to synchronized
activity in the LCN and PL. Combining optogenetic and
electrophysiological methods will allow us to test the \emph{hypothesis}
that inactivation of LCN D1DR-expressing neurons results in interval
timing deficits and decreased synchronized neuronal activity in the LCN
and PL.

\textbf{\emph{Aim 2. Determine if stimulating LCN D1DR-expressing
neurons compensates for PL dysfunction }}

Cerebellar stimulation may reduce cognitive symptoms in
schizophrenia\textsuperscript{33} but it is unknown to what extent and
by which mechanisms. We can pharmacologically mimic frontal cortex
dysfunction, allowing us to determine whether optogenetic stimulation of
LCN D1DR-expressing neurons rescues interval timing performance and
impacts neuronal activity in the PL. This design will allow us to test
the \emph{hypothesis} that stimulation of LCN D1DR-expressing neurons
can compensate for aberrant activity in the PL and improve interval
timing performance. Here we will measure `compensation' based on
improvement in interval timing task performance precision and increased
frontal cortex single unit synchronization and LFP 1-4Hz power.

The proposed project will provide extensive training in vital techniques
for my future research such as, focal drug infusion, optogenetic
stimulation, electrophysiological recording, and conducting associated
data analyses. Additionally, our results will provide further insight
into cerebellar modulation of the frontal cortex during cognitive
processing at baseline and in pharmacologically altered states.
Therefore, our findings could better define therapeutic targets for
treatment of cognitive symptoms of schizophrenia.

\end{document}
