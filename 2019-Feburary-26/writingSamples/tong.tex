\PassOptionsToPackage{unicode=true}{hyperref} % options for packages loaded elsewhere
\PassOptionsToPackage{hyphens}{url}
%
\documentclass[]{article}
\usepackage{lmodern}
\usepackage{amssymb,amsmath}
\usepackage{ifxetex,ifluatex}
\usepackage{fixltx2e} % provides \textsubscript
\ifnum 0\ifxetex 1\fi\ifluatex 1\fi=0 % if pdftex
  \usepackage[T1]{fontenc}
  \usepackage[utf8]{inputenc}
  \usepackage{textcomp} % provides euro and other symbols
\else % if luatex or xelatex
  \usepackage{unicode-math}
  \defaultfontfeatures{Ligatures=TeX,Scale=MatchLowercase}
\fi
% use upquote if available, for straight quotes in verbatim environments
\IfFileExists{upquote.sty}{\usepackage{upquote}}{}
% use microtype if available
\IfFileExists{microtype.sty}{%
\usepackage[]{microtype}
\UseMicrotypeSet[protrusion]{basicmath} % disable protrusion for tt fonts
}{}
\IfFileExists{parskip.sty}{%
\usepackage{parskip}
}{% else
\setlength{\parindent}{0pt}
\setlength{\parskip}{6pt plus 2pt minus 1pt}
}
\usepackage{hyperref}
\hypersetup{
            pdfborder={0 0 0},
            breaklinks=true}
\urlstyle{same}  % don't use monospace font for urls
\setlength{\emergencystretch}{3em}  % prevent overfull lines
\providecommand{\tightlist}{%
  \setlength{\itemsep}{0pt}\setlength{\parskip}{0pt}}
\setcounter{secnumdepth}{0}
% Redefines (sub)paragraphs to behave more like sections
\ifx\paragraph\undefined\else
\let\oldparagraph\paragraph
\renewcommand{\paragraph}[1]{\oldparagraph{#1}\mbox{}}
\fi
\ifx\subparagraph\undefined\else
\let\oldsubparagraph\subparagraph
\renewcommand{\subparagraph}[1]{\oldsubparagraph{#1}\mbox{}}
\fi

% set default figure placement to htbp
\makeatletter
\def\fps@figure{htbp}
\makeatother


\date{}

\begin{document}

Hi workshop members,

My writing sample below is the last portion of my Prospectus
introduction. In the Prospectus document, I proposed to use the ABCD
dataset to build a deep learning model distinguishing children with and
without family history (FH) of Alcohol Use Disorder (AUD). In previous
sections, I've talked about what is known about anatomical and
connectivity alterations in family history positive (FHP) subjects, as
compared to family history negative (FHN) subjects. The writing sample
below is my rationale why do I choose deep learning over traditional
regression method. I have cut out many details such as description
of/steps involved in deep learning, as I will mention them in the Method
section. Please let me know if I need to further explain anything and if
my justification is lacking

I have grayed out the portion that exceeds the word limit, which I
included because I think it provides some context, i.e., how I think my
study can address the unknown in the literature. I would appreciate your
comments on this portion, but you don't have to.

Thank you!

\textbf{3.2. Strengths of the Current Study}

In the extant literature, no quantitative meta-analysis has been run to
find consistent findings of both brain morphometry and connectivity
alterations across studies, and a qualitative review paper (McPhee et al
2018) pointed out inconsistent findings on brain morphometry changes,
even for the most robust effect (reduced amygdalar volume). In addition,
previous studies on both brain morphometry and resting-state functional
connectivity had relatively small sample size (less than 100 FHP, except
for Dager et al 2015). In many studies, FHP had substance use history,
making it difficult to tease apart neural alterations associated FH
status and personal substance use.

The Adolescent Brain Cognitive Development study provides a rich dataset
from more than 11,000 children age 9-10 with minimal substance use.
Using deep learning to analyze data from substance naïve FHN and FHP
children from this dataset can elucidate changes in brain structure and
connectivity associated with FH of AUD in a large group of participants.

\textbf{3.2.1 Strengths of machine learning}

Traditional whole brain analysis for both structural and connectivity
data typically involves mass-univariate analysis, which has significant
limitations (Vieira et al 2017). Firstly, mass-univariate analysis
involves multiple independent comparisons (e.g., independently comparing
amygdalar and hippocampal volumes between FHP and FHN). This results in
stringent correction for multiple comparisons, which can increase Type
II error. In addition, multiple comparisons assume that each brain
regions are independent, which is not accurate. Vieira et al (2017) also
noted that mass-univariate techniques can have statistical inferences at
the group level, but not at the level of the individual. Diagnostic and
treatment decisions, on the other hand, are made at the individual
level.

These limitations can be addressed with machine learning, which is a
multivariate method to detect trends and patterns in existing data,
which can later be used to make predictions on new data. Machine
learning also allows for prediction to be made at the individual level.

The gap in the literature that this project hopes to address is the
inconsistent findings in brain structure and functional connectivity in
FH of AUD. Using a large sample from the ABCD database is one of the
strategies to achieve reliable results. Using machine learning is the
additional strategy to improve reliability, as machine learning can
extract replicable brain patterns that distinguish FHP from FHN. In
particular, machine learning algorithms can pick up the overall patterns
of a dataset while ignoring noisy, idiosyncratic characteristics that
are specific to that dataset, increasing generalizability and
replicability of the results.

\textbf{3.2.1 Strengths of deep learning}

One criticism of conventional machine learning techniques is not
performing well on raw data, especially high dimensional brain imaging
data, necessitating the ``feature selection'' step, which involves
discarding redundant information to focus solely on informative
features. Feature selection is oftentimes subjective and requires
extensive prior knowledge deducing what features should be considered
informative.

Deep learning, on the other hand, is more objective as it can learn
features from raw data, bypassing the ``feature selection'' step.
Inconsistent findings from previous studies on FH of AUD demonstrated
the subtle and diffuse structural and connectivity abnormalities
associated with family history status. Deep learning can outperform
traditional machine learning in datasets with such subtle and diffuse
patterns (LeCun et al., 2015). In addition, unlike univariate methods,
deep learning automatically takes into account the inter-correlation
between features and down-weight unmeaningful interactions.

Indeed, several studies have demonstrated deep learning's superior
performance relative to other machine learning methods (Koyamada et al.
2015; Plis et al. 2014). For instance, using Deep Neural Networks to
classify brain states associated 7 different tasks (Emotion, Gambling,
Language, Motor, Relational, Social and Working Memory) from 499
subjects, Koyamada et al. (2015) found better performance for a Deep
Neural Networks model (mean accuracy of 50.74\%) compared to supervised
learning methods (mean accuracy of 47.97\%) such as Linear Regression
and Support Vector Machine. In a different study, using a Deep Belief
Network model to classify T1-weighted structural data from 198
schizophrenic patients and 191 controls, Plis et al. (2014) achieved
90\% classification accuracy with deep learning, while a Support Vector
Machine model only achieved 68\% classification accuracy.

\end{document}
