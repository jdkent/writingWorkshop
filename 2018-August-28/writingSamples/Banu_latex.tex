\documentclass[]{article}
\usepackage{lmodern}
\usepackage{amssymb,amsmath}
\usepackage{ifxetex,ifluatex}
\usepackage{fixltx2e} % provides \textsubscript
\ifnum 0\ifxetex 1\fi\ifluatex 1\fi=0 % if pdftex
  \usepackage[T1]{fontenc}
  \usepackage[utf8]{inputenc}
\else % if luatex or xelatex
  \ifxetex
    \usepackage{mathspec}
  \else
    \usepackage{fontspec}
  \fi
  \defaultfontfeatures{Ligatures=TeX,Scale=MatchLowercase}
\fi
% use upquote if available, for straight quotes in verbatim environments
\IfFileExists{upquote.sty}{\usepackage{upquote}}{}
% use microtype if available
\IfFileExists{microtype.sty}{%
\usepackage[]{microtype}
\UseMicrotypeSet[protrusion]{basicmath} % disable protrusion for tt fonts
}{}
\PassOptionsToPackage{hyphens}{url} % url is loaded by hyperref
\usepackage[unicode=true]{hyperref}
\hypersetup{
            pdfborder={0 0 0},
            breaklinks=true}
\urlstyle{same}  % don't use monospace font for urls
\IfFileExists{parskip.sty}{%
\usepackage{parskip}
}{% else
\setlength{\parindent}{0pt}
\setlength{\parskip}{6pt plus 2pt minus 1pt}
}
\setlength{\emergencystretch}{3em}  % prevent overfull lines
\providecommand{\tightlist}{%
  \setlength{\itemsep}{0pt}\setlength{\parskip}{0pt}}
\setcounter{secnumdepth}{0}
% Redefines (sub)paragraphs to behave more like sections
\ifx\paragraph\undefined\else
\let\oldparagraph\paragraph
\renewcommand{\paragraph}[1]{\oldparagraph{#1}\mbox{}}
\fi
\ifx\subparagraph\undefined\else
\let\oldsubparagraph\subparagraph
\renewcommand{\subparagraph}[1]{\oldsubparagraph{#1}\mbox{}}
\fi

% set default figure placement to htbp
\makeatletter
\def\fps@figure{htbp}
\makeatother


\date{}

\begin{document}

\textbf{Paper title: Chronic prenatal interleukin-17 is sufficient to
cause sex-specific~ASD-phenotypes in a mouse model}

\textbf{Introduction}

Pro-inflammatory maternal immune activation (MIA) during gestation has
been associated with autism spectrum disorder (ASD) in clinical
populations (1, 2), with some further suggesting that maternal
immunological perturbations may serve a direct, causal role in its
pathogenesis (3, 4). Specifically, studies in both humans and animal
models have implicated elevated \emph{maternal} interleukin-17 (IL-17),
a key pro-inflammatory cytokine, in offspring ASD and ASD-related
phenotypes (5). The dysregulation of both T helper 17 (Th17)
lymphocytes, a subpopulation of CD4 + T cells that secrete IL-17 family
cytokines, and elevated levels of IL-17A have also been linked directly
to ASD \emph{within individuals} (6, 7), as have copy-number variants in
the \emph{IL17} gene (8). One cross-sectional study reported that nearly
50\% of children with ASD (and nearly 70\% with severe ASD) had
above-average levels of serum IL-17a (9). Interestingly, serum levels of
IL-17 have also been shown to be increased in children with ASD who
experienced symptomatic regression compared to those who did not regress
(10).

In animals, behavioral phenotypes thought to model dimensions of ASD
have previously been shown to be sensitive to broad prenatal immune
insult (11-13). More specifically, research has implicated IL-17A
signaling as necessary for the effects of gestational poly(I:C), a viral
mimetic and MIA model, on mouse offspring ASD-like phenotypes. Choi et
al. (2016) demonstrated that the selective inhibition of IL-17-producing
Th17 cells by genetic deletion of a transcription factor critical to
their development (RORyt), or maternal pre-treatment with IL-17a
blocking antibody, resulted in a rescue of MIA-associated ASD-like
behaviors, including social and repetitive/stereotyped behaviors, as
well as cortical disorganization (14). Cortical ``patches'' or
disorganized regions of cortex have also been implicated in human ASD
(15). Subsequent work by the same group further revealed that these
phenotypes may be due to hyperactivity of pyramidal neurons and
decreased GABAergic drive in the primary somatosensory cortex (16).
Critically, this work reveals that maternal mechanisms, mediated by
IL-17, may underlie ASD-like effects in MIA offspring.

IL-17 can act both via maternal systems to influence neurodevelopment,
but also directly on neurons themselves. For instance, IL-17 mechanisms
can alter cell differentiation, survival, and signaling (5). Liu et al.
(2014) demonstrated that IL-17a regulates adult hippocampal
neurogenesis, the levels of other pro- and anti-inflammatory cytokines
in the hippocampal dentate gyrus, and hippocampal electrophysiology,
such that IL-17a knockout animals exhibit increased dentate synaptic
excitability and hippocampal neurogenesis (17).

While it has been shown that IL-17 pathways mediate and are necessary
for the effects of MIA on offspring neurodevelopment and behavior, no
studies have yet examined whether chronic, maternal IL-17 throughout
pregnancy is in fact sufficient to induce these effects. To better
understand the neurodevelopmental programming role of this specific
inflammatory factor, we exposed dams to IL-17 continuously throughout
gestation. Using a combination of behavioral, genetic, and histological
approaches, we found that embryonic cortical morphogenesis and cortical
transcriptomic profiles, as well as adult neurobiology and behavior, in
male offspring are altered by prenatal maternal IL-17. This work
underscores the causal role of maternal IL-17 in the generation of
ASD-relevant phenotypes in offspring.

\end{document}
